\chapter{Test Plan}

\section{Unit Testing}
\par Unit testing of our web application was useful in the preliminary stages of testing. This form of testing can be done to ensure that we have correct functionality of individual features as well as catch any bugs related to the features. The first focus, and most important, of our testing was to carefully develop our web page to reflect our wire frame concepts. With an active web page, we were able to ensure that our website dimensions held up adequately on different browsers and varying computer operating systems. Testing our web page on varying platforms ensured the integrity of our aesthetics was kept amongst the possible user base. Another focus of this testing was to ensure our web crawler grabs the correct data we specified. The web crawler testing was tedious, but necessary, as we had to carefully prepare it for the possible items it needed to search for on-line. We were able to notice that some games were initially missed because team names were abbreviated from time to time. After adjusting the web crawler to accommodate these occurrences it was able to work successfully in grabbing the designated videos. The next function we addressed was the logging in and registration of user profiles to create a profile to use on our web page. Creating and connecting several user accounts to the firebase system allowed us to test for successful users registrations as well as logins. These tests verified that we had a user system properly set up and accessible for future integration testing. Lastly, the elasticsearch database we decide to use was also tested to ensure it can hold a high amount data entries as well as be reliable to grab entries efficiently. Once all of our video information was carefully assembled we were able to push it into the database. We used Kibana as a visualizer as it showed the successful amounts of pushed entries into our database so we could catch early on if any items were not accounted for after the transfer of information. This was a fantastic tool as it revealed to us that some game scores and/or team names were not imported into the database, which we then quickly amended solving the issue. It also allowed us to test queries to our database which we used to discover the most optimal search terms to return the most relevant information to users. 

\section{Integration Testing}
\par The integration testing of our program was one of the most crucial steps. During this testing phase we began connecting services and functionality to ensure that the program can function as a whole. Components will be pieced together one at a time so that the errors can be pinpointed and the problem can be specified. As each service is connected to the completed system the errors will become obvious as to which portion of the project is improperly functioning. We first connected our web crawler to the database. We needed to ensure that the information we grabbed was following the schema needed to be imported into the database. Although the web crawler correctly grabbed information, we realized that when the information was temporarily stored, such as the date meta data, the formatting would be distorted and unusable to the database. Adjusting the date data string format and encapsulating all the information as a JSON object allowed for the two services to work together. Now that we had our information collection coupled with our database we were able to move forward into integration with our web page. The web page integration with the previous components was focused on connecting the elasticsearch database to the web page features. We needed to ensure that the user inputs would work with the database queries. After connecting the services we discovered that the user inputs needed to match our database entries, so depending on the user input we would modify their search terms to ensure the user would get the results they requested. Our final step for integration testing was linking our web application’s features to be used with user profiles. During this portion of testing we used user profiles to select their favorite teams to be saved to their account preferences. This integration of services revealed problems in saving user preferences to a profile if they had not selected a specific team in a sport. Since the user selection for that sport was empty the data entry would not be created in the database and thus no longer accessible in the future. We were able to address this issue with a dummy variable so that the user could adjust their preferences appropriately later on.  


\section{Alpha Testing}
\par Alpha testing was completed in-house by our team once integration testing was completed. During alpha testing we were able to check that the core functionality of the product was near completion and fully functional now that all components were connected. As the system was tested, we were able to find the bugs that still persisted as well as the errors that were still prevalent in the code causing erroneous results. Initially the elasticsearch and web page were able to run simultaneously, as expected. However, playing the videos had become an issue due to the CORS being sent accessed. This issue was only noticeable once all the features were combined and running together on the network. It was resolved after modifying the proper headers and locations of the files. Along with catching the bugs in the program, we will also be able to ensure that all requirements were properly addressed and satisfied by the system. In this step we were able to reexamine our requirements to address those we felt could be further developed in our project as we had established the initial abilities we could build upon of the project. 

\section{Beta Testing}
\par During beta testing we expanded testing from only group members to friends, colleagues, or others interested in the project. By this step most user cases were thoroughly tested and functional. We used tester feedback to improve the user-interface aspects that were generally agreed upon. A few areas repeatedly mentioned were adjusting which pages were linked to specific buttons and the queue lengths. An example of this was most of our testers were either basketball fans or knew more of the sports teams from that league. This influenced our decision to show basketball team's first when customizing user preferences. Users also suggested that we limit the initial queue size so that when they search for videos they can add a greater of number of videos to the queue after a search.  As some bugs will be still uncaught before this process, using beta testers will ensure that all possible situations and edge cases are exhausted so that the program is ready to be used. 
